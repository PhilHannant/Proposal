\documentclass{article} \title{Proposal} \author{Philip Hannant} 

\begin{document} \maketitle{} \section{Introduction \& Background}

Detecting musical time is a skill which is not only a fundamental musical skill [1] but also something that can seemingly come naturally to humans, the majority being able to analyse and reproduce discrete metrical stuctures of a piece of music [2]. Producing algorithms to replicate this nature human ability is probably first attempted by Longuet-Higgins [1], where he began to consider that rhythm as a binary tree with each node representing a note or rest. This theory developed into a system which would use a static tolerence limit on how much a the downbeats varied and enabling the perceived tempo to be adjusted accordingly [add longuet references]. Since Longuet-Higgins first work there have been a number of systems developed to perform beat tracking and temporal analysis with the majority using the ``surfboard'' method first described by schloss which observed the peaks of sound energy within a piece of music in order to discern the beat locations and temporal information [schloss]. 

This method however was not considered accurate enough to fully replicate the skill of a trained musician at beat detection, it was soon recognised that onset detection would play a fundamental part in any future beat detection algorithms. Where the an indicator of a new onset is seen as ``an increase in energy (or amplitude) within some frequency band(s)'' [4].




Needs to add history of all the work carried out on audio analysis with an emphasis on tempo detection

Beat detection and tempo analysis is a fundamental skill honed by musicians and in particularly drummers, their ability to perform at a consistent tempo determines a piece of musics rhythm. 

Describe how far it has come and MIREX group

There has been a lot of work carried out in the field of tempo detection, Hugh Christopher Longuet-Higgins work probably represented the first attempt to follow beat performance in relation to tempo [1]. 

Describe current tempo detection systems for drummers in midi and how current use of tempo detection algorithms in software for djs - highlight possible implementation of a training system for drummers to perfect their timing.

http://www.roland.co.uk/blog/exploring-roland-dt-1-v-drums-tutor-software/ - such systems currently are only available for midi instruments as the timing data is recorded using midi triggers (describe triggers) so offer an very high degree of accuracy

\subsection{}

\maketitle{} \section{Aims and Objectives}
As part of my project I propose to build a real time drumbeat tempo analyser which wiill implement a variety of different wave analysis algorithms. sound energy analysis and discrete wavelet transform technologies. I will provide more details regarding the system architecture in the following sections. 

Investigate with a sole focus on drums how accurate/reliable the beatroot and discrete wavelet transform algorithms are in order to ascertain how viable a future mobile metronome and drum tempo training application would be.



\subsection{Proposed Architecture}
In order to function sufficiently the system will need to encompass the following features:

\begin{itemize}
\item Provide real time feedback to player on the tempo of the current drum beat.
\item 
\item Sufficent signal processing, the solution will need to process the audio in appropirate durations 
\end{itemize}

\subsubsection

\maketitle{} 
\section{Development Plan for the Solution}

Adapt beetroot library in order to allow for efficient real time analysis

implement matlab algorithm in scala/java

develop audio capture and processing system 

\maketitle{} 
\section{Project Schedule}



\end{document}
